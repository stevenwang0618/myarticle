\documentclass[a4paper]{article}      % The type of the document.
\usepackage{latexsym,bm,amsthm,amsmath,fancyhdr}
\setlength{\voffset}{-15.4mm}
\setlength{\textwidth}{150mm}
\title{The exact solution of plate with}        % The title and author of the document.
\author{gao}

\begin{document}

\maketitle

\section{Introduction}
The lateral deflection in plain-plate problem is described by the equilibrium equation
    \begin{equation}\label{equation 1}                                       %Equation(1)
    \frac{\partial^{4}w}{\partial x^{4}}+ 2\frac{\partial^{4}w}{\partial x^{2} \partial y^{2}}+\frac{\partial^{4}w}{\partial y^{4}}=p(x,y)
    \end{equation}
where $ p(x,y)$ is a continuously distributed lateral load.
The general solution of the partial Equation (\ref{equation 1}) is given by
    \begin{equation}\label{equation 2}                                       %Equation(2)
    w(x,y)=w_{0}(x,y)+\varphi_{1}(x+iy)+\varphi_{2}(x-iy)+x[\psi_{1}(x+iy)+\psi_{2}(x-iy)]
    \end{equation}
where $w_{0}(x,y)$ is a particular solution of Equation (\ref{equation 1}) and $\varphi_{1}$,$\varphi_{2}$,$\psi_{1}$,$\psi_{2}$ are four arbitrary functions.
\newtheorem{theorem}{Theorem}[section]
\begin{theorem}                                                             %This is the theorem
  (Four boundaries are clamped) The solution of the following plate problem
    \begin{equation}\label{equation 3}                                       %Equation(3)
     \left\{ {\begin{array}{*{20}{c}}
    {\frac{{{\partial ^4}w}}{{\partial {x^4}}} + 2\frac{{{\partial ^4}w}}{{\partial {x^2}\partial {y^2}}} + \frac{{{\partial ^4}w}}{{\partial {y^4}}} = p(x,y)}\\
    {w(x,0)=w(x,b)=\frac{{\partial w}}{{\partial y}}(x,0)=\frac{{\partial w}}{{\partial y}}(x,b)=0}\\
    {w(0,y)=w(a,y)=\frac{{\partial w}}{{\partial x}}(0,y)=\frac{{\partial w}}{{\partial x}}(a,y)=0}
    \end{array}} \right.
    \end{equation}
is given by
\end{theorem}

\begin{proof}[{\bf Proof}]%
\item[\quad Step 1.] To make the solution (\ref{equation 2}) satisfy the boundary conditions $w(x,0)=0$ and $\frac{\partial w}{\partial y}(x,0)=0$, we must put
         \begin{equation}\label{equation 4}                                 %Equation(4)
         \begin{split}
         \varphi_{1}(x)+\varphi_{2}(x)+x[\psi_{1}(x)+\psi_{2}(x)]&=A(x)\\
         \varphi_{1}'(x)-\varphi_{2}'(x)+x[\psi_{1}'(x)-\psi_{2}'(x)]&=B(x)\\
         \end{split}
         \end{equation}
    where $A(x)=-w_{0}(x,0)$ and $B(x)=i\frac{\partial w_{0}}{\partial y}(x,0)$. Letting
         \begin{equation}\label{equation 5}                                 %Equation(5)
           \varphi_{1}(x)+x\psi_{1}(x)=\alpha(x)
         \end{equation}
    where $\alpha(x)$ is an arbitrary function. By (\ref{equation 5}) and the first equation of Eqs.(\ref{equation 4}), we have
         \begin{equation}\label{equation 6}                                 %Equation(6)
          \varphi_{2}(x)+x\psi_{2}(x)=A(x)-\alpha(x)
         \end{equation}
    (\ref{equation 5}) and (\ref{equation 6}) imply that
         \begin{equation}\label{equation 7}                                 %Equation(7)
         \left\{ {\begin{array}{*{20}{l}}
         {\varphi_{1}(x)=-x\psi_{1}(x)+\alpha(x)}\\
         {\varphi_{2}(x)=-x\psi_{2}(x)+A(x)-\alpha(x)}
         \end{array}} \right.
         \end{equation}
    Substituting (\ref{equation 7}) into the second equation of Eqs. (\ref{equation 4}) yields
        \begin{equation}\label{equation 8}                                  %Equation(8)
        \psi_{2}(x)=\psi_{1}(x)-2\alpha'(x)+A_{1}(x)
        \end{equation}
    where $A_{1}(x)=A'(x)+B(x)$. By (\ref{equation 7}) and (\ref{equation 8}), we get
        \begin{equation}\label{equation 9}                                  %Equation(9)
        \left\{ {\begin{array}{*{20}{l}}
         {\varphi_{1}(x)=-x\psi_{1}(x)+\alpha(x)}\\
         {\varphi_{2}(x)=-x\psi_{1}(x)+2x\alpha'(x)-\alpha(x)+A_{2}(x)}\\
         {\psi_{2}(x)=\psi_{1}(x)-2\alpha'(x)+A_{1}(x)}
        \end{array}} \right.
        \end{equation}
    where $A_{2}(x)=A(x)-xA_{1}(x)$.%
    \item[\quad Step 2.] Similarly, to make the solution (\ref{equation 2}) satisfy the boundary conditions $w(x,b)=0$ and $\frac{{\partial w}}{{\partial y}}(x,b)=0$, we must put
        \begin{equation}\label{equation 10}                                 %Equation(10)
        \left\{ {\begin{array}{*{20}{l}}
         {\varphi_{1}(x+ib)+\varphi_{2}(x-ib)+x[\psi_{1}(x+ib)+\psi_{2}(x-ib)]=C(x)}\\
         {\varphi_{1}'(x+ib)-\varphi_{2}'(x-ib)+x[\psi_{1}'(x+ib)-\psi_{2}'(x-ib)]=D(x)}\\
        \end{array}} \right.
       \end{equation}
    where $C(x)=-w_{0}(x,b)$ and $D(x)=i\frac{{\partial w_{0}}}{{\partial y}}(x,b)$. Letting
        \begin{equation}\label{equation 11}                                 %Equation(11)
         \varphi_{1}(x+ib)=-x\psi_{1}(x+ib)+\beta(x+ib)
        \end{equation}
    where $\beta(x)$ is an arbitrary function. By (\ref{equation 11}) and the first equation of Eqs. (\ref{equation 10}), we have
    \begin{equation}\label{equation 12}                                     %Equation(12)
      \varphi_{2}(x-ib)=-x\psi_{2}(x-ib)+C(x)-\beta(x+ib)
    \end{equation}
    Using (\ref{equation 11}), (\ref{equation 12}) and the second equation of Eqs. (\ref{equation 10}), we obtain
        \begin{equation}\label{equation 13}                                     %Equation(13)
        \psi_{2}(x-ib)=\psi_{1}(x+ib)-2\beta'(x+ib)+C'(x)+D(x)
        \end{equation}
    where $C_{1}(x)=C'(x+ib)+D(x+ib)$ and $C_{2}(x)=C(x+ib)-(x+2ib)C_{1}(x)$.
    By (\ref{equation 11}), (\ref{equation 12}) and (\ref{equation 13}), we have
        \begin{equation}\label{equation 14}                                 %Equation(14)
        \left\{ {\begin{array}{*{20}{l}}
        {\varphi_{1}(x)=-(x-ib)\psi_{1}(x)+\beta(x)}\\
        {\varphi_{2}(x)=-(x+ib)\psi_{1}(x+2ib)+2(x+ib)\beta'(x+2ib)-\beta(x+2ib)+C_{2}(x)}\\
        {\psi_{2}(x)=\psi_{1}(x+2ib)-2\beta'(x+2ib)+C_{1}(x)}
        \end{array}} \right.
       \end{equation}
    By Eqs. (\ref{equation 9}) and Eqs. (\ref{equation 14}), we obtain
        \begin{equation}\label{equation 15}                                 %Equation(15)
         -x\psi_{1}(x)+\alpha(x)=-(x-ib)\psi_{1}(x)+\beta(x)
        \end{equation}
        \begin{equation}\label{equation 16}                                 %Equation(16)
        -x\psi_{1}(x)+2x\alpha'(x)-\alpha(x)+A_{2}(x)=-(x+ib)\psi_{1}(x+2ib)+2(x+ib)\beta'(x+2ib)-\beta(x+2ib)+C_{2}(x)
        \end{equation}
        \begin{equation}\label{equation 17}                                 %Equation(17)
        \psi_{1}(x)-2\alpha'(x)+A_{1}(x)=\psi_{1}(x+2ib)-2\beta'(x+2ib)+C_{1}(x)
        \end{equation}
    By (\ref{equation 15}), we get
        \begin{equation}\label{equation 18}                                 %Equation(18)
        \alpha(x)=ib\psi_{1}(x)+\beta(x)
        \end{equation}
    By $(\ref{equation 16})+x\times(\ref{equation 17})$, we have
        \begin{equation}\label{equation 19}                                 %Equation(19)
        -\alpha(x)=-ib\psi_{1}(x+2ib)+2ib\beta'(x+2ib)-\beta(x+2ib)+E(x)
        \end{equation}
    where $E(x)=C_{2}(x)+xC_{1}(x)-A_{2}(x)-xA_{1}(x)$. By $(\ref{equation 15})+(x+ib)\times(\ref{equation 16})$, we obtain
        \begin{equation}\label{equation 20}                                 %Equation(20)
        ib\psi_{1}(x)-2ib\alpha'(x)-\alpha(x)=-\beta(x+2ib)+E_{1}(x)
        \end{equation}
    where $E_{1}(x)=C_{2}(x)+(x+ib)C_{1}(x)-A_{2}(x)-(x+ib)A_{1}(x)$.
    By (\ref{equation 18}) and (\ref{equation 19}), we get
    \begin{equation}\label{equation 21}                                     %Equation(21)
    \alpha(x+2ib)-\alpha(x)=2ib\beta'(x+2ib)+E(x)
    \end{equation}
    By (\ref{equation 18}) and (\ref{equation 20}), we get
    \begin{equation}\label{equation 22}                                     %Equation(22)
    \beta(x+2ib)-\beta(x)=2ib\alpha'(x)+E_{1}(x)
    \end{equation}
    By (\ref{equation 21}) and (\ref{equation 22}), we get
    \begin{equation}\label{equation 23}                                     %Equation(23)
    (2ib)^{2}\beta''(x+2ib)=\beta(x+4ib)-2\beta(x+2ib)+\beta(x)+E_{2}(x)
    \end{equation}
    where $E_{2}(x)=E_{1}(x)-E_{1}(x+2ib)-2ibE'(x)$.
\end{proof}
\newtheorem{lemma}[theorem]{Lemma}
\begin{lemma}\label{lemma 1}
    The general solution of the function equation (\ref{equation 23}) can be written as
    \begin{equation}\label{equation 24}                                     %Equation(24)
       \begin{array}{l}
        \beta (x) = {\beta _0}(x) + \sum\limits_{n = 1}^\infty  {[{a_n}exp(\frac{{{\lambda _n}x}}{{2ib}}) + {b_n}exp( - \frac{{{\lambda _n}x}}{{2ib}}) + {c_n}exp(\frac{{{\delta _n}x}}{{2ib}}) + {d_n}exp( - \frac{{{\delta _n}x}}{{2ib}})} \\
        \qquad\qquad \qquad \qquad +{e_n}exp(\frac{{{{\bar \lambda }_n}x}}{{2ib}}) + {f_n}exp( - \frac{{{{\bar \lambda }_n}x}}{{2ib}}) + {g_n}exp(\frac{{{{\bar \delta }_n}x}}{{2ib}}) + {h_n}exp( - \frac{{{{\bar \delta }_n}x}}{{2ib}})]
        \end{array}
    \end{equation}
where
    \begin{equation}\label{equation 25}                                     %Equation(25)
    exp(\frac{\lambda}{2})-exp(-\frac{\lambda}{2})=\lambda
    \end{equation}
    \begin{equation}\label{equation 26}                                     %Equation(26)
    exp(\frac{\delta}{2})-exp(-\frac{\delta}{2})=-\delta
    \end{equation}
\end{lemma}
 By (\ref{equation 21}) and (\ref{equation 22}), we can also get
    \begin{equation}\label{equation 27}                                     %Equation(27)
    (2ib)^{2}\alpha''(x)=\alpha(x+2ib)-2\alpha(x)+\alpha(x-2ib)+E_{3}(x)
    \end{equation}
    where $E_{3}(x)=-E(x)+E(x-2ib)-2ibE_{1}'(x)$.


Similar to lemma \ref{lemma 1}, we can get the general solution of equation (\ref{equation 27})
    \begin{equation}\label{equation 28}                                     %Equation(28)
       \begin{array}{l}
        \alpha (x) = {\alpha_0}(x) + \sum\limits_{n = 1}^\infty  {[{a'_n}exp(\frac{{{\lambda _n}x}}{{2ib}}) + {b'_n}exp( - \frac{{{\lambda _n}x}}{{2ib}}) + {c'_n}exp(\frac{{{\delta _n}x}}{{2ib}}) + {d'_n}exp( - \frac{{{\delta _n}x}}{{2ib}})} \\
        \qquad\qquad \qquad \qquad +{e'_n}exp(\frac{{{{\bar \lambda }_n}x}}{{2ib}}) + {f'_n}exp( - \frac{{{{\bar \lambda }_n}x}}{{2ib}}) + {g'_n}exp(\frac{{{{\bar \delta }_n}x}}{{2ib}}) + {h'_n}exp( - \frac{{{{\bar \delta }_n}x}}{{2ib}})]
        \end{array}
    \end{equation}

Substituting (\ref{equation 24}) and (\ref{equation 28}) to equation (\ref{equation 21}) or equation(\ref{equation 22}), we can get the relationship between the coefficients in the two solutions
    \begin{equation}\label{equation 29}                                     %Equation(29)
    \begin{split}
    a'_n&=a_n exp(\frac{\lambda_n}{2})\\
    b'_n&=b_n exp(-\frac{\lambda_n}{2})\\
    c'_n&=-c_n exp(\frac{\delta_n}{2})\\
    d'_n&=-d_n exp(-\frac{\delta_n}{2})\\
    e'_n&=e_n exp(\frac{\bar \lambda_n}{2})\\
    f'_n&=f_n exp(-\frac{\bar \lambda_n}{2})\\
    g'_n&=-g_n exp(\frac{\bar \delta_n}{2})\\
    h'_n&=-h_n exp(-\frac{\bar \delta_n}{2})\\
    \end{split}
    \end{equation}

Now we get the expressions of $\alpha(x)$ and $\beta(x)$, then we substitute them back to the Equations (\ref{equation 18}) and (\ref{equation 9}), we can get the expressions of the functions $\varphi_1(x)$, $\varphi_2(x)$, $\psi_1(x)$, and $\psi_2(x)$.
        \begin{equation}\label{equation 30}                                 %Equation(30)
        \left\{ {\begin{array}{*{20}{l}}
         {\psi_{1}(x)=\frac{\alpha(x)-\beta(x)}{ib}}\\
         {\varphi_{1}(x)=-x\psi_{1}(x)+\alpha(x)}\\
         {\varphi_{2}(x)=-x\psi_{1}(x)+2x\alpha'(x)-\alpha(x)+A_{2}(x)}\\
         {\psi_{2}(x)=\psi_{1}(x)-2\alpha'(x)+A_{1}(x)}
        \end{array}} \right.
        \end{equation}
In order to express $w(x,y)$, we should rewrite the Eqs. (\ref{equation 30}) as follow:
        \begin{equation}\label{equation 31}                                 %Equation(31)
        \left\{ {\begin{array}{*{20}{l}}
         {\psi_{1}(x+iy)=\frac{\alpha(x+iy)-\beta(x+iy)}{ib}}\\
         {\varphi_{1}(x+iy)=-(x+iy)\psi_{1}(x+iy)+\alpha(x+iy)}\\
         {\varphi_{2}(x-iy)=-(x-iy)\psi_{1}(x-iy)+2(x-iy)\alpha'(x-iy)-\alpha(x-iy)+A_{2}(x-iy)}\\
         {\psi_{2}(x-iy)=\psi_{1}(x-iy)-2\alpha'(x-iy)+A_{1}(x-iy)}
        \end{array}} \right.
        \end{equation}
then we can get
    \begin{equation}\label{equation 32}                                     %Equation(32)
    \begin{split}
    w(x,y) &=w_{0}(x,y)+\varphi_{1}(x+iy)+\varphi_{2}(x-iy)+x[\psi_{1}(x+iy)+\psi_{2}(x-iy)]\\
           &=w_{0}(x,y)-(x+iy)\psi_{1}(x+iy)+\alpha(x+iy)-(x-iy)\psi_{1}(x-iy)+2(x-iy)\alpha'(x-iy)-\alpha(x-iy)\\
           &+A_{2}(x-iy)+x[\psi_{1}(x+iy)+\psi_{1}(x-iy)-2\alpha'(x-iy)+A_{1}(x-iy)]
    \end{split}
    \end{equation}
We also know that $\psi_1(x-iy)=\frac{\alpha(x-iy)-\beta(x-iy)}{ib}$ and $A_2(x-iy)=A(x-iy)-(x-iy)A_1(x-iy)$. Substitute them to equation (\ref{equation 32}), we get
    \begin{equation}\label{equation 33}                                     %Equation(33)
    \begin{split}
    w(x,y) &=w_{0}(x,y)+(-\frac{y}{b}+1)[\alpha(x+iy)-\alpha(x-iy)]+\frac{y}{b}[\beta(x+iy)-\beta(x-iy)]\\
           &-2iy \alpha'(x-iy)+A(x-iy)+iy A_1(x-iy)
    \end{split}
    \end{equation}




\end{document}
