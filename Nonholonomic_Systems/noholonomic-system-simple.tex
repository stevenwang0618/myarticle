\documentclass[preprint,11pt]{elsarticle}
%final-print formatting
%\documentclass[final,authoryear,5p]{elsarticle}
%\usepackage{mathrsfs}
%\usepackage{fancyhdr}
%\pagestyle{fancy}
%\usepackage{verbatim}
%use the grphicx package
%\usepackage{graphicx}
%\usepackage{epsfig}
%\usepackage{bbding}
%\usepackage{booktabs}
%\usepackage{multirow}b
%\usepackage{tabularx}
%The amssymb package provides various useful mathematical symbols
\usepackage{amsmath,amssymb}
%% The numcompress package shorten the last page in references.
%% `nodots' option removes dots from firstnames in references.
%\usepackage[nodots]{numcompress}


\newcommand{\mathd}{\mathrm{d}}
%%Journal ID
\journal{Physics Letters A}

\begin{document}

\begin{frontmatter}
\title{Lagrange equations of nonholonomic systems using Hamilton's principle embedded with  a class of nonlinear constraints}
\author[nudt]{Dingwen Wang\corref{cor1}}
\ead{wangdingwen0618@gmail.com}
\author[nudt]{Puyun Gao}
\ead{puyungao@hotmail.com}

\cortext[cor1]{Corresponding author, Tel:+86 15116442661}
\address[nudt]{College of Aerospace Science and Engineering, National University of Defense Technology, Changsha 410072, PR China}

%\maketitle
\begin{abstract}
  This paper derives a new kind of Lagrange equations of motion for a
  class of nonholonomic dynamical systems from Hamilton's principle. The constraints given here are generally nonlinear in the velocity.
  The validity of Hamilton's principle in nonholonomic systems is discussed.
  After the constraints are embedded into the Lagrangian of the systems, Hamilton's principle is directly used just like what we have done for the holonomic systems.
  And the transpositional rule $d \delta=\delta d$ is assumed only for those generalized coordinates whose variations are independent.
  The equations of motion derived from this approach are new.
  An example is given to show how to apply the approach.

\end{abstract}

\begin{keyword}
nonholonomic systems \sep constraints\sep Hamilton's principle\sep equation of motion
\end{keyword}
\end{frontmatter}

%%%%%%%%%%%%%%%%%%%%%%%%%%%%%%%%%%%%%%%%%%%%%%%%%%%%
%%%Main Text

\section{Introduction}

%The mystery of nonholonomic systems has not been fully revealed.


Conventionally, there are two principles, which are Hamilton's principle and d'Alembert-Lagrange's principle, to treat the dynamics of physical systems with constraints.
D'Alembert's principle claims the virtual work done by the constraint forces vanishes, while Hamilton's principle says that the actual path of a system is the path that make the action a minimum.
Both the principles lead to the same equations of motion when we confined the discussion to the holonomic systems.
But when we refer to the nonholonomic systems, unfortunately, the two principles are not consistent(\cite{cronstrom2009,flannery2005}).
The essential differences between holonomic and nonholonomic systems are widely investigated in the literature(\cite{flannery2005,pars1954,jeffreys1954,hertz1896}).
The equations of motion derived from two principles are different in form, and that their solutions differ from each other has also been proved by \cite{cronstrom2009}.
However, it is widely believed that d'Alembert-Lagrange's principle is more powerful and fundamental than Hamilton's principle for systems with nonholonomic constraints.(\cite{flannery2005,flannery2011}).


\cite{flannery2005} has ruled out the extension of Hamilton's principle for nonholonomic systems and suggests even d'Alembert-Lagrange's principle is not applicable when referring to general nonholonomic constraints.

In this paper, we discuss the validity of Hamilton's principle for nonholonomic systems.
Hamilton's principle for nonholonomic systems is always involved with multiplication rule(\cite{pars1954,jeffreys1954,hertz1896,flannery2005}).
The inconsistency between the variational action principle and the principle of d'Alembert-Lagrange ties closely with the introducing of Lagrange multipliers.
We analyze the argument given by other authors(\cite{flannery2005,cronstrom2009}) and show there is no theoretical criterion to judge which principle is more fundamental than another.
To bypass the controversy on the inconsistency due to the introduction of Lagrange multipliers, we embed the constraints into the Lagrangian and apply the Hamilton's principle directly.
Finally, we derive the equations of motion of a class of nonholonomic systems without Lagrange multipliers.

\section{Holonomic Constraints}
\subsection{D'Alembert-Lagrange's principle}
Let's consider the systems where some of the coordinates are dependent on some other coordinates, the dependence can be described by following equations
\begin{equation}
  f_i  ( q, t) = 0 , \hspace{2em} ( i = 1, \cdots , k).
  \label{holcont}
\end{equation}
The constraints can be easily written in velocity form as
\begin{equation}
\dot{f}_i ( q, t) = \sum^n_{j = 1} \frac{\partial f_i}{\partial q_j}
   \dot{q}_j + \frac{\partial f_i}{\partial t} = 0, \hspace{2em} ( i = 1, \cdots,
   k) .
\end{equation}
Then the system is confined to a time varying hyper--surface in the configuration space.

First, we are to derive the equations of motion from d'Alembert-Lagrange's principle.
Assuming the external applied forces can be derived from a generalized potential and introducing $T$ as the kinetic energy and $V$ as the potential energy, we define the Lagrangian $L(q,\dot{q};t)=T-V$.
Here the generalized coordinates are not necessary independent.
D'Alembert-Lagrange's principle gives the equations of motion
\begin{equation}
  \sum_{j = 1}^n \left[ \frac{\mathd}{\mathd t} \left( \frac{\partial
  L}{\partial \dot{q}_j} \right) - \frac{\partial L}{\partial q_j}  \right]
  \delta q_j = 0 .\label{eulerlagrange}
\end{equation}

The generalized coordinates must be consistent with the constraint \eqref{holcont}.
Then the variations $\delta q_j$ must satisfy the linear set of conditions,
\begin{equation}
  \sum_{j = 1}^n \left( \frac{\partial f_i}{\partial q_j} \right) \delta q_j =
  0. \hspace{2em} ( i = 1, \cdots, k)\label{holcontvar}
\end{equation}

Introducing $k$ Lagrange's multipliers $\lambda_i$ and subtracting the sum of all variations of the constraints \eqref{holcontvar} from the left hand side of \eqref{eulerlagrange}, we get the equation
\begin{equation}\label{dalembertlagrange}
  \sum_{j=1}^n\left[\frac{\mathd}{\mathd t} \left( \frac{\partial L}{\partial \dot{q}_j} \right)
  - \frac{\partial L}{\partial q_j} - \sum_{i = 1}^k \lambda_i \frac{\partial
  f_i}{\partial q_j}\right]\delta q_j=0 \hspace{2em}  .
\end{equation}
We know there are $n-k$ independent coordinates. Supposing the last $n-k$ coordinates are independent, we choose the functions $\lambda_i$ such that
\begin{equation}\label{eleqn1}
  \frac{\mathd}{\mathd t} \left( \frac{\partial L}{\partial \dot{q}_j} \right)
  - \frac{\partial L}{\partial q_j} - \sum_{i = 1}^k \lambda_i \frac{\partial
  f_i}{\partial q_j}=0 \hspace{2em}, ( j = 1, \cdots, k).
\end{equation}
Equation \eqref{dalembertlagrange} now reduces to
\begin{equation}
  \sum_{j=k+1}^n\left[\frac{\mathd}{\mathd t} \left( \frac{\partial L}{\partial \dot{q}_j} \right)
  - \frac{\partial L}{\partial q_j} - \sum_{i = 1}^k \lambda_i \frac{\partial
  f_i}{\partial q_j}\right]\delta q_j=0,
\end{equation}
where the variations $\delta q_j$ are all independent and arbitrary. Each term in the sum must therefore individually vanish, which gives
\begin{equation}\label{eleqn2}
  \frac{\mathd}{\mathd t} \left( \frac{\partial L}{\partial \dot{q}_j} \right)
  - \frac{\partial L}{\partial q_j} - \sum_{i = 1}^k \lambda_i \frac{\partial
  f_i}{\partial q_j}=0 \hspace{2em}, ( j = k+1, \cdots, n).
\end{equation}
Together with the original constraints equations \eqref{holcont}, we now have $n+k$ independent equations that describe the motion of the system completely.
As an extra reward, the constraints force now can be expressed as
\begin{equation}\label{conforce}
  \sum_{i = 1}^k \lambda_i \frac{\partial f_i}{\partial q_j},\hspace{2em} ( j = k+1, \cdots, n).
\end{equation}
Since the constraints $f_i$ contain no $\dot{q_j}$, we can
introduce the modified Lagrangian $\tilde{L}$ over an extended set of
coordinates $s = ( q, \lambda)$, defined by
\[ \tilde{L} : = L + \sum_{i = 1}^k \lambda_i f_i, \]
then a generalized d'Alembert-Lagrange's principle shows
\begin{equation}
  \sum_{j=1}^{n+k}\left[\frac{\mathd}{\mathd t} \left( \frac{\partial \tilde{L}}{\partial \dot{s}_j}
  \right) - \frac{\partial \tilde{L}}{\partial s_j}\right] \delta s_j = 0.
\end{equation}
On regarding all $s_j$ as independent, then
\begin{equation}
  \frac{\mathd}{\mathd t} \left( \frac{\partial \tilde{L}}{\partial \dot{s}_j}
  \right) - \frac{\partial \tilde{L}}{\partial s_j} = 0 \hspace{2em} ( j = 1,
  \cdots, n + k) . \label{modifiedeom}
\end{equation}
The first $n$ equations of \eqref{modifiedeom} reproduce exact the same equations of motion as \eqref{eleqn1} and \eqref{eleqn2}, and the last $k$ equations
reproduce the constraints \eqref{holcont}.

\subsection{Hamilton's principle}
Hamilton's principle
\begin{equation}
  \delta S = \delta \int L \mathd t = \int \sum_{j = 1}^n \left[
  \frac{\mathd}{\mathd t} \left( \frac{\partial L}{\partial \dot{q}_j} \right)
  - \frac{\partial L}{\partial q_j}  \right] \delta q_j \mathd t = 0
  \label{hamiltonprinciple}
\end{equation}
is a variational principle. Again, the generalized coordinates are not necessary independent here.
Analogue to the operation performed when using d'Alembert-Lagrange's principle, we introduce Lagrange multipliers $\lambda_i(i=1,\cdots,k)$. Then we have
\begin{equation}
 \int \sum_{j = 1}^n \left[
  \frac{\mathd}{\mathd t} \left( \frac{\partial L}{\partial \dot{q}_j} \right)
  - \frac{\partial L}{\partial q_j} - \sum_{i = 1}^k \lambda_i \frac{\partial
  f_i}{\partial q_j}\right] \delta q_j \mathd t = 0
\end{equation}
Following the same arguments when using d'Alembert-Lagrange's principle, we get the same Euler-Lagrange equation as \eqref{eleqn1} and \eqref{eleqn2}.

Of course, we can also extend Hamilton's principle. By introducing the modified Lagrangian
\begin{equation}
  L^\star = L + \sum_{i = 1}^k \lambda_i f_i,
\end{equation}
equation \eqref{hamiltonprinciple} is therefore replaced by so called Hamilton's generalized principle,
\begin{equation}\label{genhamiltonprinciple}
  \delta S^\star = \delta \int L^\star \mathd t = \int \sum_{j = 1}^n \left[
  \frac{\mathd}{\mathd t} \left( \frac{\partial L^\star}{\partial \dot{s}_j} \right)
  - \frac{\partial L^\star}{\partial s_j}  \right] \delta s_j \mathd t = 0,
\end{equation}
from which the Euler-Lagrange equations \eqref{modifiedeom} can be reproduced for the independent variations of the extended set of coordinates $s=(q,\lambda)$.


\section{Using Hamilton's principle in nonholonomic systems}\label{s3}


\subsection{Analysis on the linear nonholonomic constraints}
We have shown d'Alembert-Lagrange's principle and Hamilton's principle are equivalent.
Besides, it does not matter whether we impose the constraints before or after we take variations of the Lagrangian.
But these are true only for holonomic systems or some proper nonholonomic systems.

Consider the dynamical systems with linear nonholonomic constraints
\begin{equation}
  \sum_{j = 1}^n a_{i j} ( q ; t) \dot{q}_j + b_i ( t) = 0 \hspace{2em} ( i =
  1, \cdots, k), \label{nonholonomicconstraints}
\end{equation}
conventionally, the variations $\delta q_j$ must satisfy the following equations
\begin{equation}\label{chetaev}
  \sum_{j = 1}^n a_{i j} ( q ; t) \delta q_j = 0 \hspace{2em} ( i = 1,
   \cdots, k),
\end{equation}
which are the so called Chetaev conditions(\cite{goldenstein}).

According to d'Alembert-Lagrange's principle with Lagrange multipliers, the
equations of motion have the following form
\begin{equation}\label{linearnheom}
  \frac{\mathd}{\mathd t} \left( \frac{\partial L}{\partial \dot{q}_j}
   \right) - \frac{\partial L}{\partial q_j} = \sum_{i = 1}^k \lambda_i a_{i
   j} \hspace{2em} ( j = 1, \cdots, n) .
\end{equation}

Unfortunately, the equations derived here are different from those derived from generalized d'Alembert-Lagrange's principle and Hamilton's principle.
It was conventionally claimed that generalized Hamilton's principle is not valid in nonholonomic systems.
D'Alembert-Lagrange's principle is thought to be more powerful or fundamental than Hamilton's principle in the nonholonomic systems.
Also, though linear velocity constraints remain within the scope of d'Alembert-Lagrange's principle, there isn't a valid theory for nonholonomic constraints with a general velocity dependence(\cite{flannery2005}).

Here we analyze these claims.
To get the equations of motion, the key point is that we must derive linear conditions between the coordinates variations from the constraints (\ref{holcont}) or (\ref{nonholonomicconstraints}). If this is achieved, we can give the equations of motion from the variational principles with the help of Lagrange multipliers.

But these linear conditions in nonholonomic systems such as the Chetaev condition are more similar to axioms rather than the logic deductions from the
constraints and remain without any basic theoretical justification.
Some authors don't support these conditions(\cite{guo1994}).
It is widely believed that \eqref{chetaev} is deduced from \eqref{nonholonomicconstraints} using d'Alembert-Lagrange's principle of virtual work. Equations \eqref{nonholonomicconstraints} are recast as
\begin{equation}
  \sum_{j = 1}^n a_{i j} ( q ; t) \mathd q_j + b_i ( t) \mathd t= 0 \hspace{2em} ( i =
  1, \cdots, k)
\end{equation}
Then taking a virtual displacement which means $\mathd t=0$ and $\mathd q_j = \delta q_j$, we can get \eqref{chetaev}.
But the problem is that the calculus of variations differs completely from the calculus of differentials.
$\mathd q_j$ is a differential motion over a interval $\mathd t$, and is defined by $\mathd q_j=\dot{q}_j \mathd t$. It vanished when the interval $\mathd t$ is set to zero. In fact, expressions of the constraints on the variations are
\begin{equation}
  \sum_{j = 1}^n a_{i j}  \delta \dot{ q}_j +  \sum_{j = 1,k=1}^n \frac{\partial a_{i j} }{\partial q_k} \dot{ q}_j \delta q_k +\sum_{k=1}^n \frac{\partial b_i }{\partial q_k}\delta q_k = 0 \hspace{2em} ( i =
  1, \cdots, k),
\end{equation}
There is no connection between variations and their derivatives.
Hence, \eqref{chetaev}, which play a key role to derive the equations of motion \eqref{linearnheom}, has no solid theoretical fundament. Therefore, the linear nonholonomic system is also outside the scope of d'Alembert-Lagrange's principle. Or at least, we can say there is no theoretical way to judge which principle is more applicable for nonholonomic systems.


Following the approach developed in \cite{Gao2005}, we bypass the dispute on the validity of these conditions and derive the equations of motion for some kind of nonholonomic constraints without multipliers.

Let's reconsider the holonomic systems. If we use the holonomic constraints to
reduce the number of generalized coordinates to the minimum number $r = n -
k$(say, the first $r$ variables), then we can get the equtions of motion
without multipliers. To achieve this, we must express the dependent variables
in terms of the independent generalized coordinates from the contraints $f_i$.
Then the Lagrangian contains only the independent generalized coordinates
after substituting the dependent variables into it,
\[ \Theta ( q_1, \cdots, q_r ; \dot{q}_1, \cdots, \dot{q}_r f ; t) = L ( q_1,
   \cdots, q_n ; \dot{q}_1, \cdots, \dot{q}_n ; t) . \]
The equations of motion can be derived from Hamilton's principle
\[ \delta \int_{t_0}^{t_1} L ( q_1, \cdots, q_n ; \dot{q}_1, \cdots, \dot{q}_n
   ; t) \mathd t = \delta \int_{t_0}^{t_1} \Theta ( q_1, \cdots, q_r ;
   \dot{q}_1, \cdots, \dot{q}_r ; t) \mathd t = 0. \]
Assuming $\delta q ( t_0) = \delta q ( t_1) = 0$, we get the
equations of motion
\[ \frac{\mathd}{\mathd t} \left( \frac{\partial \Theta}{\partial \dot{q}_j}
   \right) - \frac{\partial \Theta}{\partial q_j} = 0 \hspace{2em} ( j = 1,
   \cdots, r), \]
where all the $q_j$ are independent. Here we must point out that the
constraints are \textbf{embedded}, not adjoined, into the Lagrangian
before we take variation on it.

\subsection{Equations of motion for a kind of nonholonomic constraints}
Now consider the following type of general nonholonomic constraints
\begin{equation}
  q_i = \varphi_i  ( q_{k + 1}, \cdots, q_n ; \dot{q}_{k + 1}, \cdots,
  \dot{q}_n) \hspace{2em} ( i = 1, \cdots, k), \label{constraint}
\end{equation}
where $q_{k + 1}, \cdots,$ and $q_n$ are independent variables, and
$\varphi_i$ are sufficiently differentiable.
The constraint functions may also be dependent on time explicitly, but for simplicity, we disregard this case.

Then we can get the derivatives of $q_i$,
\begin{equation}
  \dot{q}_i = \sum^n_{\mu = k + 1} \frac{\partial \varphi_i}{\partial q_{\mu}}
  \dot{q}_{\mu}  + \sum^n_{\mu = k + 1} \frac{\partial
  \varphi_i}{\partial \dot{q}_{\mu}} \ddot{q}_{\mu} \hspace{1em} ( i = 1,
  \cdots, k) . \label{der}
\end{equation}
Suppose the Lagrangian of the system before the constraints embedded is
\begin{equation}
  L = L ( q_1, \cdots, q_n ; \dot{q}_1, \cdots, \dot{q}_n ; t) .
  \label{lagrangian}
\end{equation}
Substituting (\ref{constraint}) and (\ref{der}) into (\ref{lagrangian}), we
get the Lagrangian of the system after imposing the constraints
\[ \Theta ( q_{k + 1}, \cdots, q_n ; \dot{q}_{k + 1}, \cdots, \dot{q}_n ;
   \ddot{q}_{k + 1}, \cdots, \ddot{q}_n ; t) = L ( q_1, \cdots, q_n ;
   \dot{q}_1, \cdots, \dot{q}_n ; t) . \]
We note that there are only the independent variables in the Lagrangian
function $\Theta$ now. And the Lagrangian is freed from constraints.

It is not very hard to verify the following relation
\begin{equation}\label{relation}
  \left\{\begin{array}{rcl}
  \frac{\partial \Theta}{\partial q_j} &= &\frac{\partial L}{\partial q_j} +
   \sum_{i = 1}^k \frac{\partial L}{\partial q_i} \frac{\partial
   \varphi_i}{\partial q_j} + \sum_{i = 1}^k \frac{\partial L}{\partial
   \dot{q}_i} \frac{\partial \dot{q}_i}{\partial q_j} \\[6pt]
   \frac{\partial \Theta}{\partial \dot{q}_j} &= &\frac{\partial L}{\partial
   \dot{q}_j} + \sum_{i = 1}^k \frac{\partial L}{\partial \dot{q}_i}
   \frac{\partial \dot{q}_i}{\partial \dot{q}_j} + \sum_{i = 1}^k
   \frac{\partial L}{\partial q_i} \frac{\partial \varphi_i}{\partial
   \dot{q}_j} \\[6pt]
   \frac{\partial \Theta}{\partial \ddot{q}_j} &= &\sum_{i = 1}^k \frac{\partial
   L}{\partial \dot{q}_i} \frac{\partial \dot{q}_i}{\partial \ddot{q}_j} =
   \sum^k_{i = 1} \frac{\partial L}{\partial \dot{q}_i} \frac{\partial
   \varphi_i}{\partial \dot{q}_j}
  \end{array}\right.\hspace{2em}( j = k + 1, \cdots, n).
\end{equation}

According to Hamilton principle, the real path of a system is the
path that make the action an extremum,
\begin{equation}
\delta \int_{t_0}^{t_1} L \mathd t = \delta \int_{t_0}^{t_1} \Theta ( q_{k
   + 1}, \cdots, q_n ; \dot{q}_{k + 1}, \cdots, \dot{q}_n ; \ddot{q}_{k + 1},
   \cdots_{}, \ddot{q}_n, t) \mathd t = 0,
\end{equation}
where $q_{\mu}  ( \mu = k + 1, \cdots, n)$ are independent variables.
The path that makes the action has an extremum under the constraints coincides with the path that makes the
modified action an extremum without constraints imposed.

With the assumption $\delta q ( t_0) = \delta q ( t_1) = 0$ and $\delta
\dot{q} ( t_0) = \delta \dot{q} ( t_1) = 0$ and following the rule of calculus of variation, we get
\begin{equation}
 \int_{t_0}^{t_1}  \sum_{\mu = k + 1}^n \left[ \frac{\partial
   \Theta}{\partial q_{\mu}} - \frac{\mathd}{\mathd t} \frac{\partial
   \Theta}{\partial \dot{q}_{\mu}} + \frac{\mathd^2}{\mathd t^2} \left(
   \frac{\partial \Theta}{\partial \ddot{q}_{\mu}} \right) \right] \delta
   q_{\mu} \mathd t = 0.
\end{equation}



Since the variations $\delta q_{\mu}  ( \mu = k + 1, \cdots, n)$ are independent and arbitrary, the
equations of motion for all the independent variables are given by
\begin{equation}\label{modeleqn}
  \frac{\partial \Theta}{\partial q_{\mu}} - \frac{\mathd}{\mathd t}
  \frac{\partial \Theta}{\partial \dot{q}_{\mu}} +
  \frac{\mathd^2}{\mathd t^2} \left( \frac{\partial \Theta}{\partial
  \ddot{q}_{\mu}} \right) = 0 \hspace{2em} ( \mu = k + 1, \cdots, n) .
\end{equation}

The equations are different from those derived from d'Alembert-Lagrange principle with multipliers(\cite{goldenstein,Neimark}).
The constraints are relieved, we apply Hamilton's principle directly just as what we have done for holonomic system.
Besides, we don't need  the Chetaev conditions which are the basis to derive the equations of motion by other authors.
The transpositional rule $d \delta=\delta d$ is assumed only for those generalized coordinates whose variations are independent.
The only assumption is that the independent velocity variations must vanish at end points.
Though these equations are fourth-order differential equations, they are simple and clear in form, and more important, no multiplier appears.
To solve the equations, the initial conditions should be provided. Notably, the initial conditions must be consistent with the constraints.

\subsection{Comparison to previous result}
Using the relations \eqref{relation} , we recast the equations of motion \eqref{modeleqn} into
\begin{equation}
\begin{array}{lcl}
  \frac{\partial L}{\partial q_j} + \sum_{i = 1}^k \left( \frac{\partial
   L}{\partial q_i}  \frac{\partial \varphi_i}{\partial q_j}  +
   \frac{\partial L}{\partial \dot{q_i}}  \frac{\partial \dot{q_i}}{\partial
   q_j} \right) - & &\\[12pt]
   \frac{\mathd}{\mathd t} \left( \frac{\partial L}{\partial
   \dot{q_j}} + \sum_{i = 1}^k \frac{\partial L}{\partial \dot{q_i}}
   \frac{\partial \dot{q_i}}{\partial \dot{q_j}} + \sum_{i = 1}^k
   \frac{\partial L}{\partial q_i}  \frac{\partial \varphi_i}{\partial
   \dot{q_j}} \right)
    + \frac{\mathd^2}{\mathd t^2}  \left( \sum_{i = 1}^k
   \frac{\partial L}{\partial \dot{q_i}}  \frac{\partial \varphi_i}{\partial
   \dot{q_j}} \right) &=& 0\\
   ( j = k + 1, \cdots, n). & &
\end{array}
\end{equation}

Defining $\varepsilon_i ( f) = \frac{\mathd}{\mathd t} \frac{\partial
f}{\partial \dot{q_i}} - \frac{\partial f}{\partial q_i}$, we can simplify the above equation to
\begin{equation}\label{simplifiedeeom}
\begin{array}{lcl}
  \varepsilon_j ( L) - \sum_{i = 1}^k \varepsilon_i ( L) \varepsilon_j (
   \varphi_i) - \sum_{i = 1}^k \frac{\mathd^2}{\mathd t^2} \left(
   \frac{\partial L}{\partial \dot{q_i}} \right) \frac{\partial
   \varphi_i}{\partial \dot{q_j}} + \sum_{i = 1}^k \frac{\mathd}{\mathd t}
   \left( \frac{\partial L}{\partial \dot{q_i}} \right)  \frac{\partial
   \varphi_i}{\partial q_j} &= &0 \\
   ( j = k + 1, \cdots, n). & &
   \end{array}
\end{equation}

Compared with the equations given in \cite{Gao2005},
\begin{equation}
\begin{array}{lcl}
   \varepsilon_j ( L) - \sum_{i = 1}^k \varepsilon_i ( L) \varepsilon_j (
   \varphi_i) - \sum_{i = 1}^k \frac{\mathd^2}{\mathd t^2} \left(
   \frac{\partial L}{\partial \dot{q_i}} \right) \frac{\partial
   \varphi_i}{\partial \dot{q_j}} + \sum_{i = 1}^k \frac{\mathd}{\mathd t}
   \left( \frac{\partial L}{\partial q_i} \right)  \frac{\partial
   \varphi_i}{\partial \dot{q_j}}& = &0 \\
   ( j = k + 1, \cdots, n), & &
\end{array}
\end{equation}
we found that the only difference between the two equations lies in the last term.

\subsection{Discussion}
According to \cite{flannery2005} , d'Alembert-Lagrange's principle is more fundamental than
Hamilton's principle. While Hamilton's principle is
valid only for holonomic or semiholonomic systems, d'Alembert-Lagrange's
principle is applied to holonomic systems and nonholonomic system with linear
nonholonomic constraints. For general nonholonomic constraints, even
D'Alembert's principle doesn't work.

The reasons given by the authors are that the displaced paths for nonholonomic
systems are not geometrically possible and therefore do not satisfy the
multiplication rule condition, so the variational Hamilton's principle
can not be extended to the nonholonomic domain.

The equations of constraints appropriate to the displaced pathes are
\begin{equation}
f_i ( q + \delta q ; \dot{q} + \delta \dot{q} ; t) = f_i ( q ; \dot{q} ;
   t) + \delta f_i ( q ; \dot{q} ; t).
\end{equation}
The constraints for the displaced paths change by
\begin{equation}
 \delta f_i = \sum_{j = 1}^n  \frac{\partial f_i}{\partial q_j} \delta q_j (
   t) + \sum_{j = 1}^n \frac{\partial f_i}{\partial
   \dot{q}_j} \delta \dot{q}_j ( t) .
\end{equation}
With the aid of $\delta \dot{q}_j = \mathd [ \delta q_j ( t)] / \mathd t$,
this difference is
\begin{equation}
\delta f_i = \frac{\mathd}{\mathd t} \left[ \sum_{j = 1}^n \frac{\partial
   f_i}{\partial \dot{q}_j} \delta q_j ( t) \right] - \sum_{j = 1}^n F_{i j}
   \delta q_j ( t) , \label{variation}
\end{equation}
where
\[ F_{i j} = \frac{\mathd}{\mathd t} \left( \frac{\partial f_i}{\partial
   \dot{q}_j} \right) - \frac{\partial f_i}{\partial q_j} \hspace{2em} ( j =
   1, \cdots, n) . \]


\cite{flannery2005} implicitly assumes the following Chetaev conditions
\[ \sum_{j = 1}^n \frac{\partial f_i}{\partial \dot{q}_j} \delta q_j = 0
   \hspace{2em} ( i = 1, \cdots, k) . \]
The condition for the displaced paths to be all geometrically possible is that
$f_i ( q + \delta q ; \dot{q} + \delta \dot{q} ; t) = 0$, that is $\delta f_i
= 0$ and the constraints are invariant to displacement. If $F_{i j} = 0$,
which means that the constraints are integrable in essence, then the displaced paths are geometrical possible.
Therefore D'Alembert's principle and Hamilton's principle are valid for this proper nonholonomic system now.
If $F_{i j} \neq 0$, then $\delta f_i \neq 0$ and the displaced paths are geometrical impossible which means Hamilton's principle is not applicable for this case.

However, we have analyzed these claims. Especially, we have shown the validity of Chetaev conditions is controversial and has no solid fundament, so all the conclusions based on Chetaev conditions and criticism on Hamilton's principle are questionable.

In fact, by contrast to the argument given by {\cite{flannery2005}}, it is conventionally
assumed that the displaced paths should satisfy the constraints in advance,
which means $f_i ( q + \delta q ; \dot{q} + \delta \dot{q} ; t) = 0$ is always
true, therefore $\delta f_i = f_i ( q + \delta q ; \dot{q} + \delta \dot{q} ;
t) - f_i ( q ; \dot{q} ; t)$ vanishes naturally.
In view of this, Hamilton's principle is applicable for nonholonomic systems.



\section{Example}

Consider the example of sleigh on an inclined plane (\cite{Neimark}). The equation of the nonholonomic constraints of the system is given by
\begin{equation}
  \dot{y} = \dot{x} \tan \varphi . \label{examplecon}
\end{equation}
We can also recast \eqref{examplecon} as
\begin{equation}
  \varphi = \Phi ( \dot{x}, \dot{y}) =  \arctan \frac{\dot{y}}{\dot{x}}.
  \label{examplecon1}
\end{equation}
The Lagrange function of the system can be written in the following form
\begin{equation}
  L = \frac{1}{2} m \left[ \dot{x}^2 + \dot{y}^2  + J \dot{\varphi}^2 + 2 a
  \dot{\varphi} ( \dot{y} \cos \varphi - \dot{x} \sin \varphi)\right] + m g x \sin\alpha,
\end{equation}
where $m, a, J$ and $\alpha$ are constants.
From \eqref{examplecon1}, we get the derivative of $\varphi$
\begin{equation}
  \dot{\varphi}=\frac{\dot{x} \ddot{y}-\dot{y} \ddot{x}}{\dot{x}^2+\dot{y}^2}.
\end{equation}
After imposing the constraints, the modified Lagrangian is
\begin{equation}\label{examplelagrangian}
\begin{array}{rcl}
  \Theta&=&\frac{1}{2} m \left[ \dot{x}^2 + \dot{y}^2  + J \dot{\varphi}^2\right]+ m g x \sin\alpha\\[6pt]
        &=&\frac{1}{2} m \left[ \dot{x}^2 + \dot{y}^2  + J \left( \frac{\dot{x} \ddot{y}-\dot{y} \ddot{x}}{\dot{x}^2+\dot{y}^2}\right)^2\right]+ m g x \sin\alpha.
\end{array}
\end{equation}
Now the Lagrangian $\Theta$ contains only the independent coordinates $x$ and $y$.
Then the equations of motion are
\begin{equation}\label{exampleeuler}
\left\{\begin{array}{rcl}
    \frac{\partial \Theta}{\partial x} - \frac{\mathd}{\mathd t}
  \frac{\partial \Theta}{\partial \dot{x}} +
  \frac{\mathd^2}{\mathd t^2} \left( \frac{\partial \Theta}{\partial
  \ddot{x}} \right)&=&0\\[6pt]
  \frac{\partial \Theta}{\partial y} - \frac{\mathd}{\mathd t}
  \frac{\partial \Theta}{\partial \dot{y}} +
  \frac{\mathd^2}{\mathd t^2} \left( \frac{\partial \Theta}{\partial
  \ddot{y}} \right)&=&0\\
  \end{array}\right.
\end{equation}
It is not very hard to verify
\begin{equation}
\left\{\begin{array}{ccc}
   \frac{\partial \Theta}{\partial x}= m g \sin\alpha,&\hspace{2em}&\frac{\partial \Theta}{\partial \ddot{x}}=\frac{m J \dot{y}(\dot{y}\ddot{x}-\dot{x}\ddot{y})}{(\dot{x}^2+\dot{y}^2)^2}\\[6pt]
   \frac{\partial \Theta}{\partial y}= 0 ,&\hspace{2em}& \frac{\partial \Theta}{\partial \ddot{y}}=\frac{m J \dot{x}(\dot{x}\ddot{y}-\dot{y}\ddot{x})}{(\dot{x}^2+\dot{y}^2)^2}\\
  \end{array}\right.
\end{equation}
The expressions for $\frac{\partial \Theta}{\partial \dot{x}}$ and $\frac{\partial \Theta}{\partial \dot{y}}$ are more complicated but still can be derived from \eqref{examplelagrangian}. Then all the terms in \eqref{exampleeuler} can be expressed as functions of $x, y$ and their derivatives.
Given the initial conditions which are consistent with the constraints \eqref{examplecon}, the differential equations \eqref{exampleeuler} can be solved.

Of course, we can also derived the equations of motion from \eqref{simplifiedeeom}
\begin{equation}
\left\{\begin{array}{rcl}
    \varepsilon_x (L)-\varepsilon_{\varphi}(L) \varepsilon_{x}(\Phi)-\frac{\mathd^2 }{\mathd t^2}\left(\frac{\partial L}{\partial \dot{\varphi}} \right) \frac{\partial \Phi}{\partial \dot{x}}+\frac{\mathd}{\mathd t}\left( \frac{\partial L}{\partial \dot{\varphi}}\right)\frac{\partial \Phi}{\partial x} & = &0\\[12pt]
      \varepsilon_y (L)-\varepsilon_{\varphi}(L) \varepsilon_{y}(\Phi)-\frac{\mathd^2 }{\mathd t^2}\left(\frac{\partial L}{\partial \dot{\varphi}} \right) \frac{\partial \Phi}{\partial \dot{y}}+\frac{\mathd}{\mathd t}\left( \frac{\partial L}{\partial \dot{\varphi}}\right)\frac{\partial \Phi}{\partial y} & = &0\\
  \end{array}\right.
\end{equation}
By the equation \eqref{examplecon1}, we have $\frac{\partial \Phi}{\partial x}=\frac{\partial \Phi}{\partial x}=0$. Thus the above equations can be simplified as
\begin{equation}\label{simplifiedexampleeuler}
\left\{\begin{array}{rcl}
    \varepsilon_x (L)-\varepsilon_{\varphi}(L) \frac{\mathd}{\mathd t}\left( \frac{\partial \Phi}{\partial \dot{x}}\right) -\frac{\mathd^2 }{\mathd t^2}\left(\frac{\partial L}{\partial \dot{\varphi}} \right) \frac{\partial \Phi}{\partial \dot{x}}& = &0\\[12pt]
    \varepsilon_y (L)-\varepsilon_{\varphi}(L) \frac{\mathd}{\mathd t}\left( \frac{\partial \Phi}{\partial \dot{y}}\right) -\frac{\mathd^2 }{\mathd t^2}\left(\frac{\partial L}{\partial \dot{\varphi}} \right) \frac{\partial \Phi}{\partial \dot{y}}& = &0\\
  \end{array}\right.
\end{equation}
It is not hard to verify that
\begin{equation}
 \left\{\begin{array}{rcl}
 \varepsilon_x (L)&=&m \left[ \frac{\mathd}{\mathd t}\left( \dot{x}-a \dot{\varphi} \sin\varphi\right)- g \sin\alpha\right]\\[6pt]
 \varepsilon_y (L)&=&m  \frac{\mathd}{\mathd t}\left( \dot{y}+a \dot{\varphi} \cos\varphi\right)\\[6pt]
 \varepsilon_\varphi (L)&=&m \left[ \frac{\mathd}{\mathd t}\left( J \dot{\varphi}+a \dot{y}\cos \varphi-a \dot{x} \sin\varphi\right)-a \dot{\varphi}\left(\dot{y} \sin \varphi+\dot{x} \cos \varphi \right)\right]
 \end{array}\right.
\end{equation}
and
\begin{equation}
  \frac{\partial \Phi}{\partial \dot{x}}=- \frac{\dot{y}}{\dot{x}^2+\dot{y}^2}=-\frac{\sin\varphi \cos\varphi}{\dot{x}},\hspace{2em}\frac{\partial \Phi}{\partial \dot{y}}=\frac{\cos^2 \varphi}{\dot{x}}.
\end{equation}
Together with the constraint \eqref{examplecon1}, when the proper initial conditions are provided, the motion can be determined.

\section{Conclusions}
In this article, we derive the equations of motion of the nonholonomic systems with the nonholonomic constraints \eqref{constraint} from Hamilton's principle. The idea is that the path that makes the action has an extremum under the constraints coincides with the path that makes the modified action an extremum without constraints imposed. We embed the nonholonomic constraints into Lagrangian and make the modified Lagrangian free from the constraints.
The approach needs no assumptions on the relation between coordinates variations. The formulation is simple and easy to understand.

The equations of motion \eqref{modeleqn} or \eqref{simplifiedeeom} without Lagrange multipliers are new. The equations here are slightly different from those in \cite{Gao2005}.




%\section*{Reference}
\bibliographystyle{model3-num-names}
\bibliography{nonholonomic}



\end{document}
